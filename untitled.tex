\textit{Letter to the editor} 

The first reports \cite{Happold_2016} of a possible effect of valproic acid in the outcome of patients with glioblastoma were published nearly at the same time of the first ever description of the driver role of histone mutations in a human cancer - significantly enough in pediatric and young adult patients with glioblastoma \cite{Schwartzentruber_2012}. Since then, a few retrospective reports have added clinical information that fueled the hypothesis of valproic acid effect in the survival of glioblastoma patients, via HDAC-inhibiting effect. Now, the report of Happold et al \cite{Happold_2016} may have seemingly poured cold water upon this idea. Without a shadow of doubt, Happold's report is the best evidence about this question that has been provided so far. Even though someone may build a number of well grounded critiques to their findings, this will be mostly futile given the weaknesses of the previous evidence. One particular type of bias that seems to be very common in retrospective accounts of would-be repurposed drugs is the \textit{immortal time bias} \cite{Ho_2012} (aka survivor bias). This kind of problem arises when one has to forcefully add the pre-therapy events (deaths, in this case) to the non-treated group (because such patients never had the opportunity to receive the therapy of interest, they have died before!).It is very easy and amusing to demonstrate that any such an arrangement of an observational cohort study will necessarily suggest a beneficial effect of the treatment. As Happold et al has cited \cite{Happold_2016}, our group was the one that originally published retrospective data suggesting a possible effect of valproic acid in pediatric patients with brain tumors. Our original account included a heterogeneous group of patients, but since then, we published other reports studying more homogeneous cohorts of patients, with mixed results \cite{Felix_2012}. Recently, we presented a poster in a Brazillian Society of Pediatric Oncology Congress with new data on the outcome of DIPG patients treated with a contemporary radiochemotherapy protocol and valproic acid \cite{59c42273-d778-4fe5-8019-07a0e4509517}. 