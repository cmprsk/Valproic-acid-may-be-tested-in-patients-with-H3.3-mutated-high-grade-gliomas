\textit{Letter to the editor} 

The first reports \cite{Happold_2016} of a possible effect of valproic acid in the outcome of patients with glioblastoma were published nearly at the same time as the first ever description of the driver role of histone mutations in a human cancer - significantly enough in pediatric and young adult patients with glioblastoma \cite{Schwartzentruber_2012}. Since then, a few retrospective reports have added clinical information that fueled the hypothesis of valproic acid effect in the survival of glioblastoma patients, via HDAC-inhibiting effect. Now, the report of Happold et al \cite{Happold_2016} may have seemingly poured cold water upon this idea. Without a shadow of doubt, Happold's report is the best evidence about this question that has been provided so far. Even though someone may build a number of well grounded critiques to their findings, this will be mostly futile given the weaknesses of the previous evidence. 

One particular type of bias that seems to be very common in retrospective accounts of would-be repurposed drugs is the \textit{immortal time bias} \cite{Ho_2012} (aka survivor bias). This kind of problem arises when one has to forcefully add the pre-therapy events (deaths, in this case) to the non-treated group (because such patients never had the opportunity to receive the therapy of interest, they have died before!). It is very easy and amusing to demonstrate that any such an arrangement of an observational cohort study will necessarily suggest a beneficial effect of the treatment. 

As Happold et al has cited \cite{Happold_2016}, our group was the one that originally published retrospective data suggesting a possible effect of valproic acid in pediatric patients with brain tumors. Our original account included a heterogeneous group of patients, but since then, we published other reports studying more homogeneous cohorts of patients, with mixed results \cite{Felix_2012}. Recently, we presented a poster in a Brazillian Society of Pediatric Oncology Congress with new data on the outcome of DIPG patients treated with a contemporary radiochemotherapy protocol and valproic acid \cite{59c42273-d778-4fe5-8019-07a0e4509517}. We showed a small cohort of 9 patients with DIPG treated with valproic acid throughout the observation time. This group of patients had a 6-month survival of 86\% and a 12-month survival of 64\%. While this could be regarded as reassuring and an indication for further investigation, we aknowledge that the number of patients is too small and any comparison would be severely underpowered. However, one of the mainstays of investigational clinical science (and one that is too frequently forgotten) is to gather prior information about the therapy of interest and the target cohort. Mixed results from mixed groups of patients are hardly conclusive of anything, so we must examine thoroughly the rationale for any proposed treatment. 

In this particular case, evidence has been building up that points to the dependence of a subset of pediatric high-grade gliomas from histone H3.3 (H3F3A gene) mutations \cite{Wu_2012,Bender_2013,Venneti_2014,Solomon_2015}. To be specific, pediatric patients with DIPG and midline glioblastomas harboring the H3F3A-K27M mutation comprise a subgroup with a particularly poor prognosis, compared to other patients. Mutations K27M of H3.3 and H3.1 histones define two mutually exclusive DIPG subgroups that differ in biology, behavior and prognosis \cite{Castel_2015}. The H3F3A-K27M mutant cells seem to loose methylation markers of H3.3 and undergo derepression of epigenetically silenced genes \cite{Chan_2013}. Thus, it is possible that H3F3A-K27M may render tumor cells more sensitive to HDAC inhibitors like valproic acid. We propose that future trials exploring this possibility select the patients by this molecular marker, including DIPG and glioblastoma patients with H3F3A-K27M mutation, and excluding other patients that may not benefit from the treatment.