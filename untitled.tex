\textit{Letter to the editor} 

The first reports \cite{Happold_2016} of a possible effect of valproic acid in the outcome of patients with glioblastoma were published nearly at the same time of the first ever description of the driver role of histone mutations in a human cancer - significantly enough in pediatric and young adult patients with glioblastoma \cite{Schwartzentruber_2012}. Since then, a few retrospective reports have added clinical information that fueled the hypothesis of valproic acid effect in the survival of glioblastoma patients, via HDAC-inhibiting effect. Now, the report of Happold et al \cite{Happold_2016} may have seemingly poured cold water upon this idea. Without a shadow of doubt, Happold's report is the best evidence about this question that has been provided so far. Even though someone may build a number of well grounded critiques to their findings, this will be mostly futile given the weaknesses of the previous evidence. One particualr type of bias that seems to be very common in retrospective accounts of would-be repurposed drugs is the immortal man bias 